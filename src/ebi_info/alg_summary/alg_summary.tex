\documentclass[10pt]{article}
\addtolength{\oddsidemargin}{-.875in}
\addtolength{\evensidemargin}{-.875in}
\addtolength{\textwidth}{1.75in}
\addtolength{\topmargin}{-.875in}
\addtolength{\textheight}{1.75in}
\usepackage{algorithmic}
\title{Boundary Integral Solver Algorithm Summary}

\newcommand{\cS}{{\cal S}}
\newcommand{\cD}{{\cal D}}
\newcommand{\cR}{{\cal R}}
\newcommand{\cK}{{\cal K}}

\begin{document}
\maketitle

\subsection*{Surface definition}
We start with the description of the overlapping-chart surface
abstraction used in the code, which is based on (Ying, Biros, Zorin, 2006);
however, we make some slight changes in notation, to be more consistent with the code.

The surface is viewed as an abstract domain $M$.
The overlapping charts domains $U_k$  are
equipped with bijective maps $\xi_k: P_k \rightarrow U_k$, where
$P_k$ is a subset of $M$; $U_k$ cover all of $M$, and may overlap.

			    For each chart, there is a geometry map
$g^k: U_k \rightarrow \mathbf{R}^3$, and a partition-of-unity blending
function $w_k: U_k \rightarrow \mathbf{R}$. Defining
$\tilde{w}_k(s) = w_k(\xi^{-1}_k(s))$ for any point $s \in P_k$, and zero for other
points of $M$, we require $\sum_k \tilde{w}_k(s) = 1$, for all points
$s \in M$.

In the code, there are two realizations of this abstraction. 
Both require some way of identifying points on $M$.

In the specific case of manifold-based surfaces of (Ying and Zorin
2004), the domain $M$ corresponds to an abstract (i.e., with no 3d
embedding specified) quad mesh, i.e. a collection of unit squares (faces) with
a coordinate system $(c,d)$  specified on each square with an
arbitrary vertex of the square chosen as the origin, with coordinates
along edges. The edges of the squares are identified, so that each edge is shared by exactly two
squares. Additionally, faces sharing a vertex form a loop such that
each two squares in a sequence share an edge.

In this case, the points of $M$  are identified by a pair $(F,q)$,
where $q\in [0,1] \times [0,1]$, and $F$ is the index of the face.
To maintain uniqueness,  the edges and vertices are arbitrarily associated with one of
the faces they belong to (this is never explicitly used int the code,
as all surface samples are in the interior of faces).

A patch $P_k \subset M$, consists of all quads $F_i$  sharing a vertex
$V_k$. The map $\xi_k$ and its inverse are analytically defined on
each face of $P_k$, as $\xi_k(z) = z^{4/K}|z^{1-4/K}|$, where $K$ is the vertex
valence, and the  coordinate system $(c,d)$  on $F_i$ is chosen with origin at
$V_k$,  with $z = c +id$. This is a conformal map rescaled in the radial direction to reduce
distortion.  $\xi_k(P_k)$ is a curved star-shaped domain in the
plane; the coordinates in the plane are denoted $(x,y)$ in the code. 

The correspondence between the notation described here and the code is
$\xi_k$ is \verb@face_point_to_xy@, $\xi_k^{-1}$
is \verb@xy_to_face_point@, $g^k$ is \verb@xy_to_patch_coords@ (also
can compute derivatives), $w_k$ is \verb@xy_to_patch_value@ (the last two terms should be
changed -- these are misleading).


\subsection*{Notation}

\begin{itemize}
\item $d$ is the dimension of the density in the problem, $1$ or $3$ (vector density).
\item $\delta$  is the target sampling step in 3d;
per-patch parametric domain sampling step $h_k$  is computed from $\delta$;
\item $R$ is the refinement factor; for the points in $\Omega_1$, refined
sampling is used  used with 3d spacing $\delta/R$, $R > 1$.				      
\item $\rho$ is the radius, in the number of patch-depedent
parametric domain step sizes, of the floating partition of unity $\eta$ used
in singular evaluation.
\item $L$ is the number of sample points used for near evaluation interpolation. 				      
\item  A star-shaped patch $U_k$ has zero-centered bounding square
$D_k$ of size $2b_k \times 2b_k$.
\item \emph{sample points} $s_{k,n}$ are associated with patches and are picked
from a regular grid on $D_k$; $s_{k,n} = \xi_k^{-1}( -b_k + i h_k,
-b_k + j h_k)$, for some integers $(i,j)$.  Points of $D_k \setminus U_k$ are
skipped.  
\item \emph{collocation points} $c_{k,m}$  are on-the-surface points
at which the integrals are evaluated  (may or may not be sample
points: these are sample points in the solve, but can be any for
solution  evaluation); for a patch $P_k$, the list of collocation points includes both
points sampled on $D_k$, as well as all points sampled on chart
bounding boxes $D_\ell$, for which $P_k \cap P_\ell \neq \emptyset$.

\item Each patch $U_k$ is associated with a connected component of the boundary, $m(k)$, which has an
  interior point $z_m$.
 
\item $\cS, \cD, \cR$ are single-layer, double-layer, and rotlet-type kernel
for a given equation, the latter beyinf defined only for 
\item  $ r^\times $ denotes the  linear operator of taking cross product with $r$. 
\item $\phi$ denotes density values;
\item $u$ denotes potentials in evaluation (in solve, the output is also a density).
\end{itemize}


\subsection*{Algorithms} 

\paragraph{Setup.}


\begin{algorithmic}
\STATE\COMMENT{{\bf Sampling setup} }
\STATE\COMMENT{ {\tt DN3dOv::setup()}}
\FORALL{patches $P_k$}
\STATE Estimated per-patch scale $\sigma_k :=  max(|g^k_x(0,0)|, |g^k_y(0,0)|)$.
\STATE Grid resolution $n_k := \lfloor 2 b_k/(\delta/\sigma_k) \rfloor$.
\STATE Parametric  domain step $h_k := 2b_k/n_k$;
\STATE Create the set of sample points $s_{k,i}$ on $P_k$, $i=1\ldots N_k$,  which are images of the subset of the  $n_k \times n_k$ grid points on $D_k$ inside $U_k$.   
\STATE\COMMENT{Each point also has a global index $n$, and denoted $s_n$ when a global index is used.}
\FORALL{$s_{k,i}$, $i=1\ldots N_k$}
\STATE Compute: 
\STATE $p_{k,i} := g_k(s_{k,i})$, 3d position;
\STATE $n_{k,i}$, unit normal to the surface, pointing in the direction of the domain;
\STATE $w_{k,i}$, value of the  POU function, supported on  $U_k$; 
\STATE $J_{k,i}$, determinant of the Jacobian of $g_k$;
\STATE $q_{k,i} := h_k^2$, integration weight of the point (periodic trapezoidal);
\STATE $W_{k,i} := w_{k,i}J_{k,i}q_{k_i}$, combined weight for summation.		   
\ENDFOR
\ENDFOR
\end{algorithmic}

\begin{algorithmic}
\STATE\COMMENT{{\bf Collocation setup}}
\STATE \COMMENT{{\tt DN3dOv::distribute\_collocation\_points}}
\STATE Input: list of target surface samples $t_{k,i}$, at which the integrals are evaluated (not necessarily $s_{k,i}$)
\STATE Output: collocation points $c_{k,m}$
\FORALL{patches $P_k$}
\FORALL{patches $P_\ell$ overlapping $P_k$}
\STATE Add all sample points $s_{\ell,i}$ in  $P_\ell \cap P_k$ to the list $c_{k,j}$.
\STATE $p^c_{k,j} := g^k(c_{k,j})$;
\STATE Define $S(k,j) := (\ell,i)$, the map from collocation point on
patch $k$ to its original sample point on patch $\ell$.
\ENDFOR
\ENDFOR
\end{algorithmic} 
\newpage

\begin{algorithmic}
\STATE\COMMENT{{\bf Solver setup} }
\STATE\COMMENT{ {\tt Bis3dOvGeneric::setup()}}
\STATE Call sampling setup for spacings $\delta$ and $\delta/R$, to get samples $s_{k,i}$, and $s^r_{k,i}$. 
\COMMENT{The refined sampling is needed only for near evaluation, not described yet}
\STATE Call collocation setup for $t_{k,i}$ set to $s_{k,i}$ \COMMENT{This is called in{\tt BE3dRon::setup()}}
\STATE For bounded-domain problems, $M$ is the number of interior connected components of the boundary;
\STATE For unbounded-domain problems, $M$ is the total number of boundary components. 
\IF{$d=1$} 
\STATE {Define $N \times M$  matrix $B$, $B_{nm} = \cS(z_m, s_n)$,
corresponding to the pole terms in the equation for mutiply-connected bounded domains and unbounded domains.}
\STATE Define $N \times M$ constraint matrix $C$, with $C_{nm} = q_n$, the quadrature weight at $n$-th point.
\ELSE
\STATE Define $N \times 2M$  block matrix $B$, with $3 \times 3$ blocks $B_{mn} = \cS(z_m,s_n)$, for $m \leq M$, and $B_{nm} = \cR(z_m,s_n)$, $m> M$.
\STATE Define $N \times 2M$ constraint block matrix $C$ with $3 \times 3$ blocks $C_{nm} = q_n I$, for $m \leq M$,    and $C_{nm} = q_n (p_n -z_m)^\times$.
\ENDIF
  \STATE Compute $P := (C^T B)^{-1}$ (Schur complement for preconditioning).
\end{algorithmic}

Next, we describe the matvec for the GMRES solve, assuming the setup above.  The main component is evaluation of the singular integral
$\int \cD(x,y) \phi(y) dy$ over the surface.

\paragraph{Solve.}
The integral equation solved depends on the domain (see paper). The input is the righ-hand side $b$, defined at the sample points
  $s_{k,i}$, which is passed to the KSP solver. The complete vector we solve for is  $x = [\phi, \gamma]$,  where
$\phi$ is the double-layer density defined at $s_{k,i}$, and  $\gamma = \alpha$ for $d=1$, and $\gamma = [\alpha,\beta]$ for $d=3$.
  and $\alpha$ and $\beta$  are vectors of pole charges defined at $z_m$, (``translational'' and ``rotational'' in 3d), of size $dM$. 

\begin{algorithmic}
\STATE \COMMENT{{\bf  Singular integral evaluation}}
\STATE Input: $\phi$ at $s_{k,i}$, collocation points $c_{k,j}$.  Output: potential $u$ at target samples $t_{k,i}$. 
\STATE \COMMENT{{\tt BE3dOvRon::singular\_evaluation}}
\STATE Set weighted density $\psi_n := W_n \phi_n$;
\STATE Call FMM to evaluate $u_n = \sum_m \cK(p_m,p_n)\psi_m$ for all sample points $s_n$.
\COMMENT{subtract the part of the integral FMM computed inaccurately}
\FORALL{patch $P_k$}
\FORALL{collocation points $c_{k,j}$}
\STATE Find all sample points $s_{k,i}$, such that  $|\xi_k(s_{k,i})- \xi_k(c_{k,j})| < h_k\rho$
\STATE  Here the collocation-to-samples map is used to update correct entries in density 
\STATE $ u_{S(k,j)} := u_{S(k,j)} - \sum_i \cK(p^c_{k,j},p_{k,i}) W_{k,i} \phi_{k,i} \eta(s_{k,i})$.
\ENDFOR
\ENDFOR
\STATE \COMMENT{Add  an accurate calculation of the same part using polar samples}
\FORALL{patch $U_k$}
\FORALL{collocation point $c_{k,m}$}
\STATE Compute a radial sampling pattern $t_{k,m,i}$  in $U_k$, centered at
$\xi_k(c_{k,m})$, radius $h_k \rho$, with radial number of samples
$[\rho]$, and angular $2[\rho]$.
\STATE Interpolate weights, normals and densities to $t_{k,m,i}$ in $U_k$.
\STATE $u_{S(k,m)} := u_{S(k,m)} + \sum_i \cK(p^c_{k,m},g^k(t_{k,q})) W_{k,m,i} \phi_{k,m,i} \eta(t_{k,m,i})$.
\ENDFOR
\ENDFOR
\end{algorithmic}

For complete evaluation, the contribution from poles  need to be          
added (using $B$), and constraint equation part of the matrix
applied (using $C$). Additionally, for incompressible equations we add an extra
summation with normal-derived kernel.

\newpage
											
\begin{algorithmic}
\STATE\COMMENT{{\bf Full matvec for the Krylov solver iteration.}}
\STATE\COMMENT{{\tt Bis3dOvGeneric::mmult()}}
\STATE Input: $x = [\phi, \gamma]$,  Output: $y= [\phi^y, \gamma^y]$.
\STATE Compute $\phi^y := \frac{1}{2}\phi + \int \cD(x,y) \phi(y) dy $ using singular evaluation above. 
\IF{incompressibility constraint is present}
\STATE $\phi^y_m :=\phi^y_m+  W_m (n_m \dot \phi_m)n_m$
\ENDIF
\IF{$M > 0$ }
\STATE $\phi^y := \phi^y + B\gamma$; $\gamma^y := C^T\phi$.
\ENDIF
\end{algorithmic}

The final component of the solve is the preconditioner application,
which is used only if $M > 0$. 

\begin{algorithmic}
\STATE\COMMENT{{\bf Preconditioner application.}}
\STATE\COMMENT{{\tt Bis3dOvGeneric::pcmult()}}
\STATE\COMMENT{{\tt Supposed to agree with Greengard, Kropinski and Mayo}}
\STATE 	 $\gamma^y := -\frac{1}{2}P(\gamma - 2 C^T \phi)$
\STATE   $\phi^y  := 2\phi + B\gamma^y$          
\end{algorithmic}

\paragraph{Evaluation.}
Given the extended density $[\phi, \gamma]$, evaluate the solution at
arbitrary points in the domain, with contribution from $\gamma$ added
directly. 

The density integral for far points ($\Omega_2$ domain, greater than $C_1
\delta $ from the boundary), FMM is used.

For on-the-surface samples $t_{k,i}$,  compute collocation points $c_{k,m}$ and call 
singular integral evaluation above.

The remaining evaluation type is near evaluation

\begin{algorithmic}
\STATE\COMMENT{{\bf Near evaluation}}
\STATE Input: list of target 3d positions $p^t_m$ closer than $C\delta$ to the
surface, list of surface samples  $s^t_m$ with 3d positions $p^{t,c}_m$ closest to $p^t_m$,
  in-out flags for each target point; 
  surface density $\phi_n$ at suface samples $s_n$. Output: values $u^t_m$ of the integral of $\phi$ at $p^t_m$.
  \STATE \COMMENT{ {\tt BE3dOv:setup()} and {\tt eval()}}
\STATE Separate $p^t$ into $p^{t,0}$, points closer than $\delta$ to the surface ($\Omega_0$), and $p^{t,1}$ being the rest($\Omega_1$).
\STATE Generate  a set of new target points $p^{t,0}_{m,q}$ for each $p^{t,0}_m$, $p^{t,0}_{m,q} := p^{t,c}_m + q n^{t,c}_m \delta$, $q = 1\ldots L-1$.
\STATE The union of $p^{t,1}$ and $p^{t,0}_{m,q}$ is $p^{t,int}$														  
\STATE Create collocation points $c^r_{k,n}$ for refined samples $s^r_{k,n}$
\STATE \COMMENT{these were initialized in the solver setup -- should probably be moved here}
\STATE Create collocation points $c^t_{k,n}$ for $p^{t,c}_m$.
\STATE\COMMENT{Interpolate $\phi$ to finer samples; this is done first by patch to all collocation points in the patch, then for each sample point values obtained on different patches are averaged with blending function as weights}						      
\STATE Refine $\phi_{k,i}$  to obtain values $\phi^r_{k,i}$ at  $s^r_{k,i}$.
\FORALL{$P_k$}
\FORALL{$c^r_{k,j}$}
\STATE Interpolate $w_{k,i}\phi_{k,i}$ to samples $\xi_k(c^r_{k,j})$ to obtain values $\psi^{r,c}_{k,j}$
\ENDFOR
\ENDFOR
\FORALL{$s^r_{k,j}$}
\STATE $\phi^r_{k,j} := \sum_{(\ell,m): S(\ell,m) = (k,j)} \psi^{r,c}_{\ell,m}$ \COMMENT{This is done through scatter}
\ENDFOR
\STATE Set weighted density $\psi^r_n := W^r_n \phi^r_n$;
\STATE Call FMM to evaluate $u^{t,int}_\ell := \sum_m \cK(p^r_m,p^{t,int}_\ell)\psi^r_m$ for all targets in $\Omega_1$
  \STATE \COMMENT{The above yields $u^{t,1}_m$.}
\STATE $u^{c}_m :=$ singular evaluation with $\phi$ as input at $s^t_m$.
\STATE $u^{jmp}_m :=$  jump evaluation with $\phi$ as input at $s^t_m$.
\FORALL{target points $p^{t,0}_m$}
\STATE $u^{surf}_m := \pm\frac{1}{2} u^{jmp}_m  + u^c_m$ \COMMENT{the sign is set based on in-out flag for the point}
\STATE interpolate $u^{surf}_m$ and $u^{t,0}_{m,q}$, $q=1 \ldots L-1$ using 1d Lagrangian interpolation to $p^{t,0}_m$, to obtain $u^{t,0}_m$.
\ENDFOR
\end{algorithmic}

TODO:  jump evaluation, singularity cancellation in on-the-surface evaluation, parallel aspects. 
\end{document}

    
